\documentclass[a4paper, 12pt]{article}

%Ukrainian language

\usepackage[T1,T2A]{fontenc}
\usepackage[utf8]{inputenc}
\usepackage[english,ukrainian]{babel}
\usepackage{indentfirst}

%Verbatim
\usepackage{fancyvrb}

%Math
\usepackage{amsmath,amsfonts,amssymb,amsthm,mathtools} 

% Images
\usepackage{graphicx}
\usepackage{wrapfig}

%Vector graphics
\usepackage{tikz}
\usetikzlibrary{positioning}
\usetikzlibrary{patterns}

%Plots
\usepackage{pgfplots}
\pgfplotsset{compat=1.9}

%Title
\author{Демедюк Віталій}
\title{Чисельні методи\\
	   Лабораторна робота №3\\
	   Варіант №5}
\date{\today}

%Text color
\usepackage{xcolor}

%Multirow
\usepackage{multirow}

\usepackage{float}

\usepackage{bigstrut}
\usepackage{colortbl}

\usepackage[pdftex,
colorlinks,%
linkcolor=blue,citecolor=red,urlcolor=blue,
hyperindex,%
plainpages=false,%
bookmarksopen,%
bookmarksnumbered,%
unicode]{hyperref}

\begin{document}

\maketitle

\newpage
\tableofcontents

%TASK #1-----------------------------------------------------
\newpage
\section{Задача №1}

\subsection{Умова}

Методом Якобі розв’язати систему рівнянь.

\[
\begin{pmatrix}
4 & 0 & 1 & 1 \\
0 & 3 & 0 & 1 \\
1 & 0 & 2 & 0 \\
1 & 1 & 0 & 5
\end{pmatrix}
\begin{pmatrix}
x_1 \\
x_2 \\
x_3 \\
x_4 
\end{pmatrix}
=
\begin{pmatrix}
11 \\
10 \\
7 \\
23 
\end{pmatrix}
\]

\subsection{Теоретичні відомості}

Припустімо, що діагональні коефіціенти невиродженої матриці $A$ ненульові $(a_{ii} \neq 0)$. Якщо деякі $a_{ii} = 0$, то цього можна досягти, переставивши деякі  рядки чи стовпці матриці. розділивши \textit{і-те} рівняння на $a_{ii}$, отримаємо таку СЛАР:

\[
x_i = - \sum_{j=1}^{i-1} \frac{a_{ij}}{a_{ii}}x_j - \sum_{j=i+1}^{n} \frac{a_{ij}}{a_{ii}}x_j + \frac{b_i}{a_{ii}}, i = \overline{1, n}.
\]

Задамо якесь початкове наближення $\overline{x}^0 = (x_{1}^{0},...,x_{n}^{0})$. Наступні наближення обчислимо за формулами:

\[
x_{i}^{k+1} = - \sum_{j=1}^{i-1} \frac{a_{ij}}{a_{ii}}x_j^k - \sum_{j=i+1}^{n} \frac{a_{ij}}{a_{ii}}x_j^k + \frac{b_i}{a_{ii}}, i = \overline{1, n}, k = 0, 1, ...
\]

Метод збігається, тобто $\lim\limits_{\substack{k \to \infty}} \Vert \overline{x}^k - \overline{x}\Vert = 0$, якщо виконуються умови діагональної переваги матриці $A$:  $|a_{ii}| \geqslant \sum\limits_{\substack{j = 1 \\ j \neq i}}^{n}|a_{ij}|, i = \overline{1, n}$. Якщо ж виконуються нерівності $q|a_{ii}| \geqslant \sum\limits_{\substack{j = 1 \\ j \neq i}}^{n}|a_{ij}|, i = \overline{1, n}, q < 1$, то правдива така оцінка точності:

\[
\Vert \overline{x}^k - \overline{x}\Vert \leqslant \frac{q^k}{1-q}\Vert \overline{x}^0 - \overline{x}^11\Vert
\]

Ітерації виконують, поки не буде отримано потрібну кількість цифр в компонентах розв'язку чи до виконання умови $\frac{q^k}{1-q} < \varepsilon$

Вибір останньої умови пояснюється тим, що в разі її виконання для $\overline{x}^0 = 0$ моємо оцінку

\[
\delta(\overline{x}) \leqslant \frac{\Vert\overline{x}^k - \overline{x}\Vert}{\Vert\overline{x}\Vert}  \leqslant \frac{q^k}{1-q} < \varepsilon
\]

\subsection{Необхідні обчислення}

Перевіряємо, чи виконується умова діагональної переваги нашої матриці.

\[ a_{11} \geqslant a_{12} + a_{13} + a_{14}\]
\[ 4 \geqslant 2 = 0 + 1 + 1\]
\[ a_{22} \geqslant a_{21} + a_{23} + a_{24}\]
\[ 3 \geqslant 1 = 0 + 0 + 1\]
\[ a_{33} \geqslant a_{31} + a_{32} + a_{34}\]
\[ 2 \geqslant 1 = 1 + 0 + 0\]
\[ a_{44} \geqslant a_{41} + a_{42} + a_{43}\]
\[ 5 \geqslant 2 = 1 + 1 + 0\]

Умова виконується!

\subsection{Результат роботи програми}

Вивід програми:

\begin{Verbatim}[numbers=left,xleftmargin=20mm]
Jacobi method
4x_1 + x_3 + x_4 = 11
3x_2 + x_4 = 10
1x_1 + 2x_3 = 7
1x_1 + x_2 + 5x_4 = 23
Result:
x_1 = 1 , x_2 = 2 , x_3 = 3 , x_4 = 4.
\end{Verbatim}


%TASK #2-----------------------------------------------------
\newpage
\section{Задача №2}

Методом Зейделя розв’язати систему рівнянь.

\subsection{Умова}

\[
\begin{pmatrix}
3 & 0 & 0 & 1 \\
0 & 6 & 2 & 0 \\
0 & 2 & 3 & 0 \\
1 & 0 & 0 & 4
\end{pmatrix}
\begin{pmatrix}
x_1 \\
x_2 \\
x_3 \\
x_4 
\end{pmatrix}
=
\begin{pmatrix}
7 \\
18 \\
13 \\
17 
\end{pmatrix}
\]

\subsection{Теоретичні відомості}

Якщо в формулі методу Якобі обчислення наступного приближення використати вже відомі нові значення $x_j^{k+1}$, $j=\overline{1, i-1}$, то отримаємо формулу

\[
x_{i}^{k+1} = - \sum_{j=1}^{i-1} \frac{a_{ij}}{a_{ii}}x_j^{k+1} - \sum_{j=i+1}^{n} \frac{a_{ij}}{a_{ii}}x_j^k + \frac{b_i}{a_{ii}}, i = \overline{1, n}, k = 0, 1, ...
\]

Достатні умови збіжностім методу Зеделя такі самі, як і для методу Якобі. Крім того, метод Зейделя збігається, якщо $A^T = A \geq 0$. Умова невід'ємності симетричної матриці $A$ означає, що невід'ємні її голові мінори.

Замінивши порядок обчислення компонент, отримаємо ще одну формулу методу Зейделя:

\[
x_{i}^{k+1} = - \sum_{j=1}^{i-1} \frac{a_{ij}}{a_{ii}}x_j^{k} - \sum_{j=i+1}^{n} \frac{a_{ij}}{a_{ii}}x_j^{k+1} + \frac{b_i}{a_{ii}}, i = \overline{1, n}, k = 0, 1, ...
\]

\subsection{Необхідні обчислення}

Перевіряємо, чи виконується умова діагональної переваги нашої матриці.

\[ a_{11} \geqslant a_{12} + a_{13} + a_{14}\]
\[ 3 \geqslant 1 = 0 + 0+ 1\]
\[ a_{22} \geqslant a_{21} + a_{23} + a_{24}\]
\[ 6 \geqslant 2 = 0 + 2 + 1\]
\[ a_{33} \geqslant a_{31} + a_{32} + a_{34}\]
\[ 3 \geqslant 2 = 0 + 2 + 0\]
\[ a_{44} \geqslant a_{41} + a_{42} + a_{43}\]
\[ 4 \geqslant 1 = 1 + 0 + 0\]

Умова виконується!

\subsection{Результат роботи програми}

Вивід програми:

\begin{Verbatim}[numbers=left,xleftmargin=20mm]
Seidel method
3x_1 + x_4 = 7
6x_2 + 2x_3 = 18
2x_2 + 3x_3 = 13
1x_1 + 4x_4 = 17
Result:
x_1 = 1 , x_2 = 2 , x_3 = 3 , x_4 = 4.
\end{Verbatim}

\end{document}
