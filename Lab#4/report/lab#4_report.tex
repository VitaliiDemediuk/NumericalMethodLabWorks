\documentclass[a4paper, 12pt]{article}

%Ukrainian language

\usepackage[T1,T2A]{fontenc}
\usepackage[utf8]{inputenc}
\usepackage[english,ukrainian]{babel}
\usepackage{indentfirst}

%Verbatim
\usepackage{fancyvrb}

%Math
\usepackage{amsmath,amsfonts,amssymb,amsthm,mathtools}

% Images
\usepackage{graphicx}
\usepackage{wrapfig}

%Vector graphics
\usepackage{tikz}
\usetikzlibrary{positioning}
\usetikzlibrary{patterns}

%Plots
\usepackage{pgfplots}
\pgfplotsset{compat=1.9}

%Title
\author{Демедюк Віталій}
\title{Чисельні методи\\
	   Лабораторна робота №4\\
	   Варіант №5}
\date{\today}

%Text color
\usepackage{xcolor}

%Multirow
\usepackage{multirow}

\usepackage{float}

\usepackage{bigstrut}
\usepackage{colortbl}

\usepackage[pdftex,
colorlinks,%
linkcolor=blue,citecolor=red,urlcolor=blue,
hyperindex,%
plainpages=false,%
bookmarksopen,%
bookmarksnumbered,%
unicode]{hyperref}

% For || ||
\usepackage{physics}

\begin{document}

\maketitle

\newpage
\tableofcontents

%TASK #1-----------------------------------------------------
\newpage
\section{Задача №1}

\subsection{Умова}

Формулою трапецій знайти інтеграл з точністю $0.001$:

\[ \int_{0}^{2.5}(x^4+2x^2+x)\,dx\]

\subsection{Теоретичні відомості}

Якщо у квадратурній формулі замкненого типу взяти $n = 1$, то отримаємо формулу трапеції.

\[ \int_{a}^{b}f(x)\,dx \approx \frac{f(a)+f(b)}{2}(b-a), \]

з оцінкою залишкового члена:

\[ |R(f)| \leq \frac{M_2(b-a)^3}{12} \]

Тоді складена формула з оцінкою залишкового члена:

\[ \int_{a}^{b}f(x)\,dx \approx h(\frac{f(x_0)}{2} + f(x_1) + \cdots + f(x_{n-1}) + \frac{f(x_n)}{2}))), \]

\[ |R(f)| \leq \frac{M_2(b-a)h^2}{12} \]

Алгебраїчний степінь точності квадратурної формули дорівнює 1. \\ Порядок точності складеної формули трапеції -- 2, а на одному проміжку -- 3.

\subsection{Необхідні обчислення}

\[ \varepsilon = 0.001 \]

\[ f(x) = x^4+2x^2+x \]

\[ f''(x) = 12x^2 + 4\ \]

\[ M_2 = \max_{x \in [0; 2.5]}|12*x^2 + 4| = 79\]

\[ h \leq \sqrt{\frac{12\varepsilon}{M_2(b-a)}} = \sqrt{\frac{12 \cdot 0.001}{79 \cdot 2.5}} \approx 0.00779 \text{ -- оцінка кроку}\]

\[ \left \lceil \frac{b-a}{h} \right \rceil = \left \lceil \frac{2.5}{0.0779} \right \rceil = 321 = n \text{ -- кількість інтервалів} \]

Візьмемо із запасом $n = 50$.

\subsection{Результат роботи програми}

Вивід програми:

\begin{Verbatim}[numbers=left,xleftmargin=20mm]
Trapezoidal rule:
Integral of x^4 + 2*x^2 + x from 0 to 2.5 equal 33.0733
\end{Verbatim}


%TASK #2-----------------------------------------------------
\newpage
\section{Задача №2}

\subsection{Умова}

Методом простої ітерації розв'язати систему рівнянь:

\[
\begin{cases}
\sin(x) + 2y = 1.6, \\
\cos(y-1) = 1;
\end{cases}
\]

\subsection{Теоретичні відомості}

\textbf{Метод Ньютона.}

Ліанерізуючи рівняння $\overline{F}(\overline{x})=0$ в околі наближення до розв'язку $\overline{x}$, отримаємо систему лінійних рівнянь відносно нового наближення $\overline{x}^{k+1}$:

\[
\overline{F}\left(\overline{x}^{k}\right)+\overline{F'}\left(\overline{x}^{k}\right)\left(\overline{{x}}^{k+1}-\overline{{x}}^{k}\right) = \overline{0}.
\]

Алгоритм розв'язання рівняння:

1) задаємо початкове наближення $\overline{x}^0$;

2) обчислимо матрицю Якобі $A_{k}\overline{z}^k = \left(\frac{\partial f_i}{\partial x_i}(\overline{x}^k)\right)_{i,j=1}^{n}$;

3) розв'язати СЛАР $A_{k}\overline{z}^k = \overline{F}\left(\overline{x}^{k}\right)$;

4) обчислити нове наближення $\overline{x}^{k+1} = \overline{x}^{k} - \overline{z}^{k}$;

5) перевірити умову $||\overline{z}^k|| < \varepsilon$; якщо її виконано, припинити процес, а ні, то повторити обчислення, починаючи з п.2).



\subsection{Необхідні обчислення}

\[
\overline{F}(\overline{x}) = 
\begin{pmatrix}
	\sin(x) + 2y - 1.6 \\
	\cos(y-1) - 1
\end{pmatrix}
\]

\[
A_k = 
\begin{pmatrix}
	\cos(x) && 2 \\
	0 && -sin(y-1)
\end{pmatrix}
\]

\[
\overline{x}^0 = (1, 0)
\]

Вивід програми:

\begin{Verbatim}[numbers=left,xleftmargin=20mm]
Solve equation system {sin(x) + 2*y - 1.6 = 0, cos(y - 1) - 1 = 0}
x = -0.411516
y = 1
\end{Verbatim}

%TASK #3-----------------------------------------------------
\newpage
\section{Задача №3}

\subsection{Умова}

Степеневим методом із точністю $10^{-3}$ знайти максимальні власні значення матриці:

\[
A = 
\begin{pmatrix}
1 & 2 & 3 \\
2 & 3 & 4 \\
3 & 4 & 5
\end{pmatrix}
\]

\subsection{Теоретичні відомості}

Алгоритм відшукання $\lambda_1$ -- найбільшого власного значення, $\overline{x}_1$ степеневим методом за формолою скалярних добутків із нормуванням $\overline{x}^n$ має такий вигляд:

1) задати $\overline{x}^0$;

2) для $k = 0, 1 \cdots$ обчислити $\overline{e}^k = \frac{\overline{x}^k}{\norm{\overline{x}^k}}, \overline{x}^{k+1} = A \overline{e}^k, \\ \mu_k = \left( \overline{x}^{k+1}, \overline{e}^k \right)$

3) продовжити процес до виконання умови $|\mu_{N} - \mu_{N-1}| < \varepsilon$, де $\varepsilon$ -- задана точність.

Тоді $\lambda_1 \approx \mu_N, \overline{x}_1 \approx \overline{e}^N$.

\subsection{Необхідні обчислення}

$\overline{x}^0 = (0, 0, 0)$

\subsection{Результат роботи програми}

Вивід програми:

\begin{Verbatim}[numbers=left,xleftmargin=20mm]
Maximal eigenvalues:
1 2 3 
2 3 4 
3 4 5 
Maximal eigenvalues equal 9.62348
\end{Verbatim}

\end{document}
