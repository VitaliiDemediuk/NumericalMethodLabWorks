\documentclass[a4paper, 12pt]{article}

%Ukrainian language

\usepackage[T1,T2A]{fontenc}
\usepackage[utf8]{inputenc}
\usepackage[english,ukrainian]{babel}
\usepackage{indentfirst}

%Verbatim
\usepackage{fancyvrb}

%Math
\usepackage{amsmath,amsfonts,amssymb,amsthm,mathtools} 

% Images
\usepackage{graphicx}
\usepackage{wrapfig}

%Vector graphics
\usepackage{tikz}
\usetikzlibrary{positioning}
\usetikzlibrary{patterns}

%Plots
\usepackage{pgfplots}
\pgfplotsset{compat=1.9}

%Title
\author{Демедюк Віталій}
\title{Чисельні методи\\
	   Лабораторна робота №1\\
	   Варіант №5}
\date{\today}

%Text color
\usepackage{xcolor}

%Multirow
\usepackage{multirow}

\usepackage{float}

\usepackage{bigstrut}
\usepackage{colortbl}

\usepackage[pdftex,
colorlinks,%
linkcolor=blue,citecolor=red,urlcolor=blue,
hyperindex,%
plainpages=false,%
bookmarksopen,%
bookmarksnumbered,%
unicode]{hyperref}

\begin{document}

\maketitle

\newpage
\tableofcontents

%TASK #1-----------------------------------------------------
\newpage
\section{Задача №1}

\subsection{Умова}

Знайти мінімальний від’ємний розв’язок $ x^3 - 6x^2 + 5x + 12 = 0 $
методом релаксації.

\subsection{Теоретичні відомості}

	Якщо в методі простої ітерації вибрати $\Psi(x) = \tau = const$, то ми отримаємо метод релаксації, формула якого має вигляд $x_{n+1} = x_n + \tau f(x_n)$, $n = 0, 1, 2, \cdots$

Цей метод збігається, якщо $-2 < \tau f'(x) < 0$.

Якщо в якомусь околі кореня виконуються умови $f'(x)<0$,\\ $0<m_1<|f'(x)|<M_1$, то метод релаксації збігається для $\tau \epsilon \left( 0; \frac{2}{M_1}{} \right)$.

Збіжність найкраща за умови:

\[ \tau = \tau_{\text{опт}} = \frac{2}{m_1 + M_1} \]

З такого вибору $\tau$ для похибки $z_n = x_n - x^*$ правдива оцінка \\
$|z_n|<q^n|z_0|, n = 0, 1, 2, \cdots$, де $q = \frac{M_1-m_1}{M_1+m_1}$.

Кількість ітерацій, які потрібно виконати для відшукання розв'язку з точністю $\varepsilon$, можна визначити з нерівності:

\[ n \geqslant \left[ \frac{\ln \left( \frac{|z_0|}{\varepsilon} \right) }{\ln \left( \frac{1}{q}  \right)} \right] + 1 \]

Якщо виконується умова $f'(x)>0$, то формулу ітераційного методу потрібно записати у вигляді $x_{n+1} = x_n - \tau f(x_n)$


\subsection{Графік функції}

\begin{tikzpicture}
\begin{axis}[
	title = {$ f(x) = x^3 - 6x^2 + 5x + 12 $},
	xlabel = {$x$},
	ylabel = {$f(x)$},
	domain = -2:7,
	xmin = -2,
	xmax = 6,
	ymin = -30,
	ymax = 30,
	grid = major,
	minor tick num = 0,
	samples=100,
	xtick={-2, -1, 0, 1, 2, 3, 4, 5, 6},
	ytick={-30, -20, -10, 0, 10, 20, 30}
]
\addplot[blue] { x^3 - 6*x^2 + 5*x + 12 };
\end{axis}
\end{tikzpicture}

\subsection{Необхідні обчислення}

$f'(x) = (x^3 - 6x^2 + 5x + 12)' = 3x^2 - 12x + 5$

На графіку функції $f(x)$ бачимо, що рівняння має 3 розв'язки. Перший в околі $\left( -2 , 0 \right)$, другий в околі $\left( 2 , 3.5 \right)$ та третій в околі $\left( 3.6 , 5 \right)$.\\

Перший корінь: в околі $\left( -2 , 0 \right)$ $f'(x) > 0$ та $0 < 5 < |f'(x)| < 41$. За формулою:
\[ \tau = \tau_{\text{опт}} = -\frac{2}{5 + 41} = -\frac{2}{46} \approx -0.043 \].
Виберемо $x_0 = 0 $. \\

Другий корінь: в околі $\left( 2 , 3.5 \right)$ $f'(x) < 0$ та $0 < 0.25 < |f'(x)| < 7$. За формулою:
\[ \tau = \tau_{\text{опт}} = \frac{2}{0.25 + 7} = \frac{2}{7.25} \approx 0.28 \].
Виберемо $x_0 = 2 $.\\

Третій корінь: в околі $\left( 3.6 , 5 \right)$ $f'(x) > 0$ та $0 < 0.68 < |f'(x)| < 20$. За формулою:
\[ \tau = \tau_{\text{опт}} = -\frac{2}{0.68 + 20} = -\frac{2}{20.68} \approx -0.097 \].
Виберемо $x_0 = 5 $.

\subsection{Результат роботи програми}

Вивід програми:

\begin{Verbatim}[numbers=left,xleftmargin=20mm]
x*0 = -1
x*1 = 3
x*2 = 4
Min negative solution: -1
\end{Verbatim}

Лог-файл:

\begin{Verbatim}[numbers=left,xleftmargin=20mm]
x0 = 0
x1 = -0.516000
x2 = -0.846458
x3 = -0.969536
x4 = -0.995377
x5 = -0.999345
x6 = -0.999908
x7 = -0.999987
x8 = -0.999998
x9 = -1.000000
x10 = -1.000000
------------------------------
x0 = 2.000000
x1 = 3.680000
x2 = 3.394857
x3 = 3.100821
x4 = 2.996727
x5 = 3.000402
x6 = 2.999952
x7 = 3.000006
x8 = 2.999999
x9 = 3.000000
------------------------------
x0 = 5.000000
x1 = 3.836000
x2 = 3.900314
x3 = 3.942975
x4 = 3.968757
x5 = 3.983345
x6 = 3.991262
x7 = 3.995455
x8 = 3.997647
x9 = 3.998785
x10 = 3.999374
x11 = 3.999677
x12 = 3.999834
x13 = 3.999914
x14 = 3.999956
x15 = 3.999977
x16 = 3.999988
x17 = 3.999994
x18 = 3.999997
x19 = 3.999998
x20 = 3.999999
x21 = 4.000000
x22 = 4.000000
x23 = 4.000000
\end{Verbatim}

%TASK #2-----------------------------------------------------
\newpage
\section{Задача №2}

\subsection{Умова}

Знайти максимальний додатний розв’язок $ x^3 + 3x^2 - x - 3 = 0 $ методом Ньютона

\subsection{Теоретичні відомості}

Метод Ньютона застосовують для розв'язання задачі $f(x)=0$ із неперервно диференційованою функцією $f(x)$. Спочатку вибирають початкове наближення $x_0$, а наступні наближення обчислюють за формулою:

\[ x_{n+1} = x_n - \frac{f(x_n)}{f'(x_n)}, n = 0, 1, 2, \cdots, f'(x_n)\neq 0 \]

Якщо $ f(x) \epsilon C^2[a; b], f(a)f(b)<0$, a $f''(x)$ не змінює знак на $[a; b]$, то для $x_0 \epsilon [a; b]$, що задовільняє умові $f(x_0)f''(x_0)>0$, можна методом Ньютона обчислити єдиний корінь рівняння із будь-яким степенем точності.

\subsection{Графік функції}

\begin{tikzpicture}
\begin{axis}[
	title = {$ f(x) = x^3 + 3x^2 - x - 3 $},
	xlabel = {$x$},
	ylabel = {$f(x)$},
	domain = -5:3,
	xmin = -5,
	xmax = 3,
	ymin = -30,
	ymax = 30,
	grid = major,
	minor tick num = 0,
	samples=100,
	xtick={-5, -4, -3, -2, -1, 0, 1, 2, 3},
	ytick={-30, -20, -10, 0, 10, 20, 30}
]
\addplot[blue] { x^3 + 3*x^2 - x - 3 };
\end{axis}
\end{tikzpicture}

\subsection{Необхідні обчислення}

Знайдемо першу і другу похідну $f(x)$:

\[ f'(x) = 3x^2 + 6x - 1 \]
\[ f''(x) = 6x + 6 \]

На графіку функції $f(x)$ бачимо, що рівняння має 3 розв'язки. Перший на проміжку $\left[-4; -2\right]$, другий на проміжку $\left[-2; 0\right]$ та третій в проміжку $\left[0; 3\right]$.\\

Перший проміжок: $\left[a; b\right] = \left[-4; -2\right]$, $f(x) \epsilon C^2\left[-4; -2\right], f(-4)f(-2) < 0$, та $f''(x)$ не змінює знак на $\left[-4; -2\right]$. Виберемо $ x_0 = -3.5$, $x_0 \epsilon \left[-4; -2\right]$ та $f(x_0)f''(x_0) > 0$, отже методом Ньютона можна обчислити єдиниий корінь рівняння.

Другий проміжок: $\left[a; b\right] = \left[-2; 0\right]$, $f(x) \epsilon C^2\left[-2; 0\right], f(-2)f(0) < 0$, та $f''(x)$ не змінює знак на $\left[-2; 0\right]$. Виберемо $ x_0 = -1.5$.

Третій проміжок: $\left[a; b\right] = \left[0; 3\right]$, $f(x) \epsilon C^2\left[0; 3\right], f(0)f(3) < 0$, та $f''(x)$ не змінює знак на $\left[0; 3\right]$. Виберемо $ x_0 = 2.5$, $x_0 \epsilon \left[0; 3\right]$ та $f(x_0)f''(x_0) > 0$, отже методом Ньютона можна обчислити єдиниий корінь рівняння.

\subsection{Результат роботи програми}

Вивід програми:

\begin{Verbatim}[numbers=left,xleftmargin=20mm]
x*0 = -3
x*1 = -1
x*2 = 1
Max positive solution: 1
\end{Verbatim}

Лог-файл:

\begin{Verbatim}[numbers=left,xleftmargin=20mm]
x0 = -3.5
x1 = -3.118644
x2 = -3.009275
x3 = -3.000064
x4 = -3.000000
------------------------------
x0 = -1.500000
x1 = -0.923077
x2 = -1.000229
x3 = -1.000000
------------------------------
x0 = 2.500000
x1 = 1.618321
x2 = 1.167004
x3 = 1.017512
x4 = 1.000225
x5 = 1.000000
\end{Verbatim}

%TASK #3-----------------------------------------------------
\newpage
\section{Задача №3}

\subsection{Умова}

Знайти максимальний додатний розв’язок $ x^3 + x^2 - 4x - 4 = 0 $ методом січних.

\subsection{Теоретичні відомості}

У методі Ньютона основна обчислювальна робота полягає у відшуканні значень $f(x)$ та $f'(x)$. Замінивши похідну $f'(x)$, використовувану в методі Ньютона, різницею послідовних значень функції, віднесеною до різниці значень аргументу(тобто замінивши дотичну січною), отримаємо таку ітераційну формулу для розв'язання рівняння $f(x) = 0$:

\[ x_{n+1} = x_n - \frac{(x_n-x_{n-1})f(x_n)}{f(x_n)-f(x_{n-1})}, n = 0, 1, 2, \cdots \]

\subsection{Графік функції}

\begin{tikzpicture}
\begin{axis}[
	title = {$ f(x) = x^3 + x^2 - 4x - 4 $},
	xlabel = {$x$},
	ylabel = {$f(x)$},
	domain = -4:4,
	xmin = -4,
	xmax = 4,
	ymin = -30,
	ymax = 30,
	grid = major,
	minor tick num = 0,
	samples=100,
	xtick={-4, -3, -2, -1, 0, 1, 2, 3, 4},
	ytick={-30, -20, -10, 0, 10, 20, 30}
]
\addplot[blue] { x^3 + x^2 - 4*x - 4 };
\end{axis}
\end{tikzpicture}

\subsection{Необхідні обчислення}

На графіку функції $f(x)$ бачимо, що рівняння має 3 розв'язки. Перший на проміжку $\left[-4; -2.5\right]$, другий на проміжку $\left[-2.5; 1\right]$ та третій в проміжку $\left[1; 3\right]$.\\

Перший проміжок: Виберемо $x_0 = -4$, $ x_1 = -3.5$

Другий проміжок: Виберемо $x_0 = 0.5$, $ x_1 = 0$

Третій проміжок: Виберемо $x_0 = 1$, $ x_1 = 1.5$

\subsection{Результат роботи програми}

Вивід програми:

\begin{Verbatim}[numbers=left,xleftmargin=20mm]
x*0 = -2
x*1 = -1
x*2 = 2
Max positive solution: 2
\end{Verbatim}

Лог-файл:

\begin{Verbatim}[numbers=left,xleftmargin=20mm]
------------------------------
x0 = -4
x1 = -3.5
x1 = -2.829268
x2 = -2.459801
x3 = -2.204508
x4 = -2.069454
x5 = -2.013794
x6 = -2.001101
x7 = -2.000019
x8 = -2.000000
------------------------------
x0 = 0.500000
x1 = 0.000000
x1 = -1.230769
x2 = -1.076433
x3 = -0.982374
x4 = -1.000964
x5 = -1.000011
x6 = -1.000000
------------------------------
x0 = 1.000000
x1 = 1.500000
x1 = 2.846154
x2 = 1.792330
x3 = 1.921289
x4 = 2.010908
x5 = 1.999484
x6 = 1.999997
x7 = 2.000000
\end{Verbatim}

\end{document}
