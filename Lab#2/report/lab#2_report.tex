\documentclass[a4paper, 12pt]{article}

%Ukrainian language

\usepackage[T1,T2A]{fontenc}
\usepackage[utf8]{inputenc}
\usepackage[english,ukrainian]{babel}
\usepackage{indentfirst}

%Verbatim
\usepackage{fancyvrb}

%Math
\usepackage{amsmath,amsfonts,amssymb,amsthm,mathtools} 

% Images
\usepackage{graphicx}
\usepackage{wrapfig}

%Vector graphics
\usepackage{tikz}
\usetikzlibrary{positioning}
\usetikzlibrary{patterns}

%Plots
\usepackage{pgfplots}
\pgfplotsset{compat=1.9}

%Title
\author{Демедюк Віталій}
\title{Чисельні методи\\
	   Лабораторна робота №2\\
	   Варіант №5}
\date{\today}

%Text color
\usepackage{xcolor}

%Multirow
\usepackage{multirow}

\usepackage{float}

\usepackage{bigstrut}
\usepackage{colortbl}

\usepackage[pdftex,
colorlinks,%
linkcolor=blue,citecolor=red,urlcolor=blue,
hyperindex,%
plainpages=false,%
bookmarksopen,%
bookmarksnumbered,%
unicode]{hyperref}

\begin{document}

\maketitle

\newpage
\tableofcontents

%TASK #1-----------------------------------------------------
\newpage
\section{Задача №1}

\subsection{Умова}

Методом Гаусса розв’язати систему рівнянь.

\[
\begin{pmatrix}
5 & 2 & 1 & 0\\
1 & 3 & 2 & 8\\
4 & -6 & 1 & 0\\
5 & 0 & 3 & 2
\end{pmatrix}
\begin{pmatrix}
x_1\\
x_2\\
x_3\\
x_4
\end{pmatrix}
=
\begin{pmatrix}
14\\
65\\
-3\\
32
\end{pmatrix}
\]

\subsection{Теоретичні відомості}

Розглянемо задачу:

\[ Ax = b \]

де $A$ - матриця з розмірності $n \times n$, $\vec{x} = (x_1, x_2, ..., x_n)^T$ -- шуканий вектор, $\vec{b} = (b_1, b_2, ..., b_n)^T$ -- заданий вектор правих частин. Припустимо, що $detA \neq 0$, тобто існує єдиний розв'язок.

Запишемо рівняння у вигляді

\[\begin{cases}
a_{11}x_1 + a_{12}x_2 + ... + a_{1n}x_n = b_1\\
a_{21}x_1 + a_{22}x_2 + ... + a_{2n}x_n = b_2\\
..................................................\\
a_{n1}x_1 + a_{n2}x_2 + ... + a_{nn}x_n = b_n\\
\end{cases}\]

Перший крок методу Гаусса (його ще називають матодом виключення невідомих) полягає у виключені невідомого $x_1$ з усіх рівнянь починаючи з другого, тобто в переході до системи 

\[\begin{cases}
x_1 + a_{12}^{(1)}x_2 + ... + a_{1n}^{(1)}x_n = b_1^{(1)}\\
 \hspace{0.9cm} a_{22}^{(1)}x_2 + ... + a_{2n}^{(1)}x_n = b_2^{(1)}\\
................................................\\
 \hspace{0.9cm} a_{n2}^{(1)}x_2 + ... + a_{nn}^{(1)}x_n = b_n^{(1)}\\
\end{cases}\]

Продовжуючи цей процес виключення, отримаємо СЛАР з верхньотрикутною матрицею вигляду

\[\begin{cases}
x_1 + a_{12}^{(1)}x_2 + ... + a_{1n}^{(1)}x_n = b_1^{(1)}\\
 \hspace{1.55cm} x_2 + ... + a_{2n}^{(2)}x_n = b_2^{(2)}\\
................................................\\
 \hspace{3.95cm} x_n = b_n^{(n)}
\end{cases}\]

Коефіціенти системи обчислюють за формулами

\[ 
a_{kj}^{(k)} = \frac{a_{kj}^{(k-1)}}{a_{kk}^{(k-1)}}, b_{k}^{(k)} = \frac{b_{k}^{(k-1)}}{a_{kk}^{(k-1)}}, k = \overline{1, n}, j = \overline{k+1, n};
\]

\[
a_{ij}^{(k)} = a_{ij}^{(k-1)} - a_{ik}^{(k-1)}a_{kj}^{(k)}, b_{i}^{(k)} = b_{i}^{(k-1)} - a_{ik}^{(k-1)}b_{k}^{(k)}, k = \overline{1, n-1}, i = \overline{k+1, n}, j = \overline{k+1, n}.
\]

за умови $a_{kk}^{(k-1)} \neq 0$.

СЛАР з верхньотрикутною матрицею можна розв'язати за формулами

\[
x_n = b_{n}^{(n)}, x_i = b_{i}^{(i)} - \sum_{j = i+1}^{n}a_{ij}^{(i)}x_j, i = \overline{n-1, 1}
\]

\subsection{Необхідні обчислення}

\[
detA = 
\begin{vmatrix}
5 & 2 & 1 & 0\\
1 & 3 & 2 & 8\\
4 & -6 & 1 & 0\\
5 & 0 & 3 & 2
\end{vmatrix}
= -450
\]

$det A \neq 0$, отже система має єдиний розв'язок.

\subsection{Результат роботи програми}

Вивід програми:

\begin{Verbatim}[numbers=left,xleftmargin=20mm]
Gaussian elimination
5x_1 + 2x_2 + x_3 = 14
1x_1 + 3x_2 + 2x_3 + 8x_4 = 65
4x_1 - 6x_2 + x_3 = -3
5x_1 + 3x_3 + 2x_4 = 32
Result:
x_1 = 1 , x_2 = 2 , x_3 = 5 , x_4 = 6.
\end{Verbatim}

%TASK #2-----------------------------------------------------
\newpage
\section{Задача №2}

\subsection{Умова}

Методом прогонки розв’язати систему рівнянь.

\[
\begin{pmatrix}
2 & 4 & 0\\
4 & 1 & 5\\
0 & 5 & 2
\end{pmatrix}
\begin{pmatrix}
x_1\\
x_2\\
x_3
\end{pmatrix}
=
\begin{pmatrix}
20\\
37\\
30
\end{pmatrix}
\]

\subsection{Теоретичні відомості}

Метод прогонки дозволяє розв'язувати СЛАР з тридіагональною матрицею. Система має такий вигляд

\[
\begin{cases}
b_1x_1 + c_1x_2 = d_1 \\
a_ix_{i-1} + b_ix_i + c_ix_{i+1} = d_i, i = \overline{2, n-1} \\
a_nx_{n-1} + b_nx_n = d_n
\end{cases}
\]

У матричному вигляді система має такий вигляд:

\[
\begin{pmatrix}
b_1 & c_1 & 0 & 0 & ....... & 0 & 0 & 0 \\
a_2 & b_2 & c_2 & 0 & ....... & 0 & 0 & 0 \\
0 & a_2 & b_2 & c_2 & ....... & 0 & 0 & 0 \\
. & . & . & . & ....... & . & . & .\\
0 & 0 & 0 & 0 & ....... & a_{n-1} & b_{n-1} & c_{n-1}  \\
0 & 0 & 0 & 0 & ....... & 0 & a_{n} & b_{n}
\end{pmatrix}
\begin{pmatrix}
x_1 \\
x_2 \\
x_3 \\
... \\
x_{n-1} \\
x_n \\
\end{pmatrix}
=
\begin{pmatrix}
d_1 \\
d_2 \\
d_3 \\
... \\
d_{n-1} \\
d_n \\
\end{pmatrix}
\]

Розв'язок проводиться в два кроки, як і в методі Гауса, прямому, та зворотному. В прямому ході ми обчислюємо:

\[
c_i^\prime = 
\begin{dcases}
\frac{c_i}{b_i}; i = 1\\
\frac{c_i}{b_i-a_ic_{i-1}^\prime}; i = \overline{2,n-1}
\end{dcases}
\]

\[
d_i^\prime = 
\begin{dcases}
\frac{d_i}{b_i}; i = 1\\
\frac{d_i-a_id_{i-1}^\prime}{b_i-a_ic_{i-1}^\prime}; i = \overline{2,n-1}
\end{dcases}
\]

\newpage
Тепер розв'язок знаходимо зворотнім ходом:

\[
x_n = d_n^\prime
\]

\[
x_i = \frac{d_i - c_ix_{i+1}}{b_i}
\]

\subsection{Необхідні обчислення}

\[
detA = 
\begin{vmatrix}
2 & 4 & 0\\
4 & 1 & 5\\
0 & 5 & 2
\end{vmatrix}
= -78
\]

$det A \neq 0$, отже система має єдиний розв'язок.

\subsection{Результат роботи програми}

Вивід програми:

\begin{Verbatim}[numbers=left,xleftmargin=20mm]
Tridiagonal matrix algorithm
2x_1 + 4x_2 = 18
4x_1 + x_2 + 5x_3 = 33
5x_2 + 2x_3 = 30
Result:
x_1 = 1 , x_2 = 4 , x_3 = 5.
\end{Verbatim}

\end{document}
